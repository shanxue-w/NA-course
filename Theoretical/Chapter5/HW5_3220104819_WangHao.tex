\documentclass[a4paper]{article}
\usepackage[affil-it]{authblk}
\usepackage{ctex}
\usepackage{amsmath}
\usepackage{amssymb}
\usepackage{amsfonts}
\usepackage{amsthm}
\usepackage{physics}
\usepackage{graphicx}
\usepackage{hyperref}
\usepackage{listings}
\usepackage{color}
\usepackage{float}
\usepackage{subcaption}
\usepackage{caption}
\usepackage{tikz}
\usepackage{pgfplots}
\hypersetup{colorlinks=true, linkcolor=blue, citecolor=blue, urlcolor=blue}


\usepackage{geometry}
\geometry{margin=1.5cm, vmargin={0pt,1cm}}
\setlength{\topmargin}{-1cm}
\setlength{\paperheight}{29.7cm}
\setlength{\textheight}{25.3cm}

\begin{document}

\title{Numerical Analysis homework \# 5}

\author{王昊 Wang Hao 3220104819
  \thanks{Electronic address: \texttt{3220104819@zju.edu.cn}}}
\affil{(Mathematics and Applied Mathematics 2201), Zhejiang University }


\date{Due time: \today}

\maketitle

\section*{I. Prove Thm 5.7}

proof: Obviously $\mathcal{C} [a,b]$ is a vector space. We need to prove it is an inner product space.

\subsection*{1. Well definedness of the inner product}

In a closed interval $[a,b]$, we have $\abs{u(t)} \leq M_1, ~ \abs{v(t)} < M_2$, so we have $\abs{\rho(t) u(t) \overline{v(t)}} \leq M_1 M_2 \abs{\rho(t)}$. Since $\rho(t)>0$ and $\rho(t) \in \mathcal{L}[a,b]$, by Lebesgue' thm, we have $\int_a^b \abs{\rho(t) u(t) \overline{v(t)}} \mathrm{d}t < \infty$. Therefore, the inner product is well defined.

% (2) Prove $\mathcal{C} [a,b]$ is a inner product space.
\subsection*{2. Linearity of the inner product}

We have
\begin{equation}
    \begin{aligned}
        \langle u, \lambda v_1 + \mu v_2 \rangle &= \int_a^b \rho (t) u(t) \overline{(\lambda v_1 + \mu v_2)(t)} \mathrm{d}t \\
        &= \int_a^b \rho (t) u(t) \overline{(\lambda v_1(t))} \mathrm{d} t + \int_a^b \rho (t) u(t) \overline{(\mu v_2(t))} \mathrm{d} t \\
        &= \overline{\lambda} \langle u, v_1 \rangle + \overline{\mu} \langle u, v_2 \rangle,
    \end{aligned}
\end{equation}

and 
\begin{equation}
    \begin{aligned}
        \langle \lambda u_1 + \mu u_2, v \rangle &= \int_a^b \rho (t) (\lambda u_1 + \mu u_2)(t) \overline{v(t)} \mathrm{d}t \\
        &= \int_a^b \rho (t) (\lambda u_1(t)) \overline{v(t)} \mathrm{d} t + \int_a^b \rho (t) (\mu u_2(t)) \overline{v(t)} \mathrm{d} t \\
        &= \lambda \langle u_1, v \rangle + \mu \langle u_2, v \rangle.
    \end{aligned}
\end{equation}  

Therefore, $\mathcal{C} [a,b]$ is an inner product space.

\subsection*{3. Normed space}

By the definition of the norm, we have
\begin{equation}
    \norm{u}_2 = 0 \Rightarrow u(t) = 0, a.e. x \in [a,b],
\end{equation}
By the continuity of $u$, we have $u(t) = 0, \forall t \in [a,b]$. Therefore, $\norm{u}_2 = 0 \Leftrightarrow u\equiv 0$.

For linearity, we have $\norm{\alpha u}_2 = ( \int_{a}^{b} \rho(t) (\alpha u(t))^2 \,\mathrm{d}t )^{\frac{1}{2}} = \abs{\alpha} \norm{u}_2$. 

For the triangle inequality, by Cauchy-Schwarz inequality, we have
\begin{equation}
    \norm{u+v}_2 = \left( \int_{a}^{b} \rho(t) (u(t)+v(t))^2 \,\mathrm{d}t \right)^{\frac{1}{2}} \leq \left( \int_{a}^{b} \rho(t) u(t)^2 \,\mathrm{d}t \right)^{\frac{1}{2}} + \left( \int_{a}^{b} \rho(t) v(t)^2 \,\mathrm{d}t \right)^{\frac{1}{2}} = \norm{u}_2 + \norm{v}_2.
\end{equation}


\section*{II. Chebyshev Polynomials of the First Kind}

\subsection*{a. Orthogonality}

The Chebyshev polynomials have the form of 
\begin{equation}
    T_n(x) = \cos(n \arccos x), \quad x \in [-1,1].
\end{equation}

So for $n \neq m$, we have
\begin{equation}
    \begin{aligned}
        \langle T_n, T_m \rangle &= \int_{-1}^{1} \frac{1}{\sqrt{1-x^2}} T_n(x) T_m(x) \mathrm{d}x \\
        &= \int_{0}^{\pi} \cos(n \theta) \cos(m \theta) \mathrm{d} \theta \quad \theta = \cos\qty(x) \\
        &= \int_{0}^{\pi} \frac{1}{2} (\cos\qty(n+m)\theta + \cos\qty(n-m) \theta ) \,\mathrm{d}\theta \\
        &= 0.
    \end{aligned}
\end{equation}

So the Chebyshev polynomials are orthogonal.

\subsection*{b. Normalize the First Three Chebyshev Polynomials}

The first three Chebyshev polynomials are $T_0(x) = 1, T_1(x) = x, T_2(x) = 2x^2 - 1$. For they are orthogonal, we only need to calculate the norm of them.

\begin{equation}
    \norm{T_0} = \sqrt{\pi}, \quad \norm{T_1} = \sqrt{\frac{\pi}{2}}, \quad \norm{T_2} = \sqrt{\frac{\pi}{2}}.
\end{equation}

So we have
\begin{equation}
    u_1 = \frac{1}{\sqrt{\pi}}, \quad u_2 = \sqrt{\frac{2}{\pi}} x, \quad u_3 = \sqrt{\frac{2}{\pi}} (2x^2 - 1).
    \label{EQ::II.B.u123}
\end{equation}

\section*{III. Approxmiation of A Continuous Function}

\subsection*{a. Approxmiate by Fourier Expansion}

The basis repect to $\rho(x) = \frac{1}{\sqrt{1-x^2}}$ is the Chebyshev polynomials, which given in \ref{EQ::II.B.u123}, so we have
\begin{equation}
    \begin{aligned}
        f(x) &= \langle y, u_1 \rangle u_1 + \langle y, u_2 \rangle u_2 + \langle y, u_3 \rangle u_3 \\
        &= \frac{1}{\sqrt{\pi}} \langle y, 1 \rangle \frac{1}{\sqrt{\pi}} + \sqrt{\frac{2}{\pi}} \langle y, x \rangle \sqrt{\frac{2}{\pi}} x + \sqrt{\frac{2}{\pi}} \langle y, 2x^2 - 1 \rangle \sqrt{\frac{2}{\pi}} (2x^2 - 1) \\
        &= \frac{2}{\pi} - \frac{4}{3 \pi} (2x^2 - 1) \\
        &= -\frac{8}{3 \pi} x^2 + \frac{10}{3 \pi}. 
    \end{aligned}
\end{equation}


\subsection*{b. Approxmiate by normal equations}

For $u_1, u_2, u_3$ orthonormal, we have
\begin{equation}
    G(u_1, u_2, u_3) = I. 
\end{equation}

For $b$ we have
\begin{equation}
    b = \begin{pmatrix} \langle y, u_1 \rangle \\ \langle y, u_2 \rangle \\ \langle y, u_3 \rangle \end{pmatrix} = \begin{pmatrix} \frac{2}{\sqrt{\pi}} \\ 0 \\ -\frac{2 \sqrt{2}}{3 \sqrt{\pi}} \end{pmatrix}.
\end{equation}

By solving the normal equations, we have
\begin{equation}
    u(x) = -\frac{8}{3 \pi} x^2 + \frac{10}{3 \pi}. 
\end{equation}


\section*{IV. Discrete Least Square via Orthonormal Polynomials}

\subsection*{a. Normalize $(1, x, x^2)$ }

To normalize $(1, x, x^2)$, by Gram-Schmidt process, we have

$\norm{u_1} = (\sum_{i=1}^{N} 1^2) ^{\frac{1}{2}} = 2 \sqrt{3}$, $v_1 = \frac{u_1}{\norm{u_1}} = \frac{1}{2 \sqrt{3}} = \frac{\sqrt{3}}{6}$.

$\langle u_2, v_1 \rangle = \sum_{i=1}^{N} u_2(t_i) \frac{1}{2 \sqrt{3}} = 13 \sqrt{3}$, $\hat{u}_2 = u_2 - \langle u_2, v_1 \rangle v_1 = x - \frac{13}{2}$, $v_2 = \frac{x - \frac{13}{2}}{\sqrt{143}}$. 

$\langle u_3, v_1 \rangle = \frac{325}{\sqrt{3}}$, $\langle u_3, v_2 \rangle = \frac{1859}{\sqrt{143}}$, $\hat{u}_3 = x^2 - 13 x + \frac{91}{3}$, $v_3 = \frac{\sqrt{3003}}{2002}(x^2 - 13 x + \frac{91}{3}) $


\subsection*{b. Find Best Approxmiation}

We can simpily get that 
\begin{equation}
    \langle y, v_1 \rangle \approx 479.778073696579, \quad \langle y, v_2 \rangle \approx 49.25465438941768, \quad \langle y, v_3 \rangle \approx 330.3306671357499.
\end{equation}

So finally we can get the best approxmiation
\begin{equation}
    \begin{aligned}
        \varphi (x) &= \langle y, v_1 \rangle v_1 + \langle y, v_2 \rangle v_2 + \langle y, v_3 \rangle v_3 \\
         &= 9.041958041958038 x^2 -113.4265734265734 x + 385.9999999999999.
    \end{aligned}
\end{equation}

Same as the result in the book.

\subsection*{c. Which can be reused}

The way here can be reused for $y_i$, because the basis won't change, we just need to recalculate the inner product is enough. 

The way in book will need to recalculate all datas, which may be time consuming.

This method can reduce the computational cost when the $y_i$ changes. 


\section*{V. Prove Theorem 5.66 and Lemma 5.67}

\subsection*{Theorem 5.66}

We have 
\begin{equation}
    A = V\Sigma U^*, A^+ y = \sum_{j=1}^{r} \frac{1}{\sigma_j} \langle y, v_j \rangle u_j = U \Sigma^{+} V^* y.
\end{equation}

So we can also get that 
\begin{equation}
    A^+ = U \Sigma^{+} V^*,
\end{equation}

Then we have 
\begin{equation}
    A A^+ A = V\Sigma U^* U \Sigma^{+} V^* V\Sigma U^* = V\Sigma U^* = V \text{diag} \{I_r, 0\} \Sigma U^* = V\Sigma U^* = A.
\end{equation}

For PDI-2, we have
\begin{equation}
    A^+ A A^+ = U \Sigma^{+} V^* V\Sigma U^* U \Sigma^{+} V^* = U \Sigma^{+} V^* = A^+.
\end{equation}

For PDI-3, we have
\begin{equation}
    \begin{aligned}
        (A A^+)^* &= (V\Sigma U^* U \Sigma^{+} V^*)^* = V \text{diag} \{I_r, 0\} V^* = A A^+, \\
        (A^+ A)^* &= (U \Sigma^{+} V^* V\Sigma U^*)^* = U \text{diag} \{I_r, 0\} U^* = A^+ A.
    \end{aligned}
\end{equation}

Therefore, we have proved Theorem 5.66.

\subsection*{Lemma 5.67}

If A has linearly independent columns, then we have $A^* A$ is invertible. If A has linearly independent columns, then we have $A^* A$ is invertible. To prove $A^+ = (A^* A)^{-1} A^*$,
we only need to prove $A^* A A^+ = A^*$, which we have
\begin{equation}
    A^* A A^+ = U \Sigma^* V^* V \Sigma U^* U \Sigma^{+} V^* = U \Sigma V^* = A^*.
\end{equation}

For the right inverse, because $A$ has linearly independent rows, we have $A A^*$ is invertible. We only need to prove $A^+ A A^* = A^*$, which we have
\begin{equation}
    A^+ A A^* = U \Sigma^{+} V^* V \Sigma U^* U \Sigma^* V^* = U \Sigma V^* = A^*.
\end{equation}

Therefore, we have proved Lemma 5.67.

\end{document}