\documentclass[a4paper]{article}
\usepackage[affil-it]{authblk}
\usepackage{float}
\usepackage{amsmath, amssymb, amsthm, amsfonts}
\usepackage[backend=bibtex,style=numeric]{biblatex}
\usepackage{graphicx}

\usepackage{hyperref}
% 设置超链接样式
\hypersetup{
	colorlinks=true,
	linkcolor=blue,
	citecolor=blue,
	urlcolor=blue,
}

\usepackage{geometry}
\geometry{margin=1.5cm, vmargin={0pt,1cm}}
\setlength{\topmargin}{-1cm}
\setlength{\paperheight}{29.7cm}
\setlength{\textheight}{25.3cm}

\addbibresource{citation.bib}

\begin{document}
% =================================================
\title{Numerical Analysis homework \# 1}

\author{Wanghao 3220104819
  \thanks{Electronic address: \texttt{3220104819@zju.edu.cn}}}
\affil{(Mathematics and Applied Mathematics 2201), Zhejiang University }


\date{Due time: \today}

\maketitle

% \begin{abstract}
%     The abstract is not necessary for the theoretical homework, 
%     but for the programming project, 
%     you are encouraged to write one.      
% \end{abstract}


% ============================================
\section*{I. Bisection Method Interval Width and Root Proximity}

\subsection*{I-a}

We can easily determine the initial interval width:
\begin{equation}
  l_0 = 3.5 - 1.5 = 2
\end{equation}

Thus, let the interval width at the $n$-th step be $l_n$, it satisfies the following relation:
\begin{equation}
  l_n = l_0 \times 2^{-n} = 2^{1-n}
\end{equation}


\subsection*{I-b}
\label{subsection:1-b}

We let the interval to be $[a_n, b_n]$, then the midpoint $c_n = \frac{a_n + b_n}{2}$. For the root $r \in [a_n, b_n]$, the distance between $c_n$ and $r$ satisfies:

\begin{equation}
  \vert c_n - r \vert \leq \min \{r-a_n, b_n-r\} \leq \frac{b_n - a_n}{2} = \frac{l_n}{2} = 2^{-n}
\end{equation}

Therefore the supermum distance between $c_n$ and $r$ is $2^{-n}$.

\section*{II. Bisection Algorithm Accuracy Proof}
\label{sec::II}

Proof: the relative error at the $n$-th step is $\eta_n = \frac{\vert c_n - r \vert}{\vert r \vert} $. Using the similar method in \ref{subsection:1-b}, we have $\vert c_n - r \vert \leq (b_0 - a_0) 2^{-n-1}$, thus we have:

\begin{equation}
  \eta_n = \frac{\vert c_n - r \vert}{\vert r \vert} \leq \frac{(b_0 - a_0) 2^{-n-1}}{r} \leq \frac{(b_0 - a_0) 2^{-n-1}}{a_0}
\end{equation}

In order to make the relative error less than $\epsilon$, we get the inequality:
\begin{equation}
  \frac{(b_0 - a_0) 2^{-n-1}}{a_0} \leq \epsilon
  \label{eq::II-1}
\end{equation}

By solving \ref{eq::II-1}, we get the following inequality:
\begin{equation}
  n \ge \frac{\log (b_0-a_0) - \log \epsilon - \log a_0}{\log 2} - 1
\end{equation}

Thus we prove the inequality.

\section*{III. Newton's Method Iterations for Polynomial Equation}

For the polynomial $p(x) = 4x^3-2x^2+3$, so $p^{\prime} = 12x^2-4x$. Therefore, the iteration relation is:
\begin{equation}
  x_{n+1} = x_n - \frac{p(x_n)}{p^{\prime}(x_n)} = x_n - \frac{4x_n^3-2x_n^2+3}{12x_n^2-4x_n}
\end{equation}

\newpage

The iteration table is shown below:

\begin{table}[H]
  \begin{center}
    \begin{tabular}{|c|c|c|}
      \hline
      $n$ & $x_n$ & $p(x_n)$\\ \hline
      0 & -1.000000000 & -3.000000000 \\ \hline
      1 & -0.812500000 & -0.465820312 \\ \hline
      2 & -0.770804196 & -0.020137887 \\ \hline
      3 & -0.768832384 & -0.000043708 \\ \hline
      4 & -0.768828086 & -0.000000000 \\ \hline
      5 & -0.768828086 & 0.000000000 \\ \hline
      6 & -0.768828086 & 0.000000000 \\ \hline
    \end{tabular}
  \end{center}
\end{table}

\section*{IV. Convergence of another Newton's Method}

In this section, we will consider a variation of Newton's method and study its convergence properties, which is defined as follows:
\begin{equation}
  x_{n+1} = x_n - \frac{p(x_n)}{p^{\prime}(x_0)}
\end{equation}

We give some notations here: $\alpha$ is the root of the function $f(x)$, $e_n$ is the error at the $n$-th step, $e_n = \vert x_n - \alpha \vert$. 

By Taylor expansion, we have:
\begin{equation}
  \begin{aligned}
    f(x_n) = f(\alpha) + f^{\prime}(\xi) (x_n - \alpha) = f^{\prime}(\xi) (x_n - \alpha)\\
  \end{aligned}
\end{equation}

So we have:
\begin{equation}
  e_{n+1} = \vert x_{n+1} - \alpha \vert = \vert 1-\frac{f^\prime(\xi)}{f^\prime(x_0)} \vert e_n, ~~\xi \in (x_n, \alpha) ~or~ \xi \in (\alpha, x_n)
\end{equation}

Therefore $C = \vert 1-\frac{f^\prime(\xi)}{f^\prime(x_0)} \vert, ~~\xi \in (x_n, \alpha) ~or~ \xi \in (\alpha, x_n)$, $s = 1$

\section*{V. Convergence of A Certain Iterative Function}

In this section, we will consider the convergence of the following iteration:
\begin{equation}
  x_{n+1} = \tanh^{-1} (x_n), ~~x_0 \in (-\frac{\pi}{2}, \frac{\pi}{2}) 
\end{equation}

We need some properties of the function $f(x) = \tanh^{-1}(x)$. If $x>0$, then $f(x) > 0$, if $x<0$, then $f(x) < 0$. 

Moreover, when $x>0,~\tanh^{-1}(x) < x$, when $x<0,~\tanh^{-1}(x) > x$.

Therefore, if $x_0 > 0$, then $0< x_1 < x_0$. By mathematical induction, we have $0 < x_n < x_{n-1} < \cdots < x_0$. So this is a monotonically decreasing and bounded sequence, thus it converges.

If $x_0 < 0$, then $0 > x_1 > x_0$. By mathematical induction, we have $0 > x_n > x_{n-1} > \cdots > x_0$. So this is a monotonically increasing and bounded sequence, thus it converges.

If $x_0 = 0$, then $x_1 = 0, \cdots, x_n = 0$, so it converges.

In conclusion, the sequence converges for all $x_0 \in (-\frac{\pi}{2}, \frac{\pi}{2})$.

\section*{VI. Value of A Continued Fraction}

In this section, we will calculate the value of the continued fraction:
\begin{equation}
  x = \frac{1}{p + \frac{1}{p + \frac{1}{p + \cdots}}}, p>1
\end{equation}

To solve this, we define $x_1 = \frac{1}{p}, x_2 = \frac{1}{p+\frac{1}{p}}$ and so forth. Then we have a iteration relation:
\begin{equation}
  x_{n+1} = \frac{1}{p + x_n}, ~~x_0 = \frac{1}{p}
\end{equation}

So the value of the continued fraction is the limit of the sequence $\{x_n\}$. Next, we will prove the convergence of the sequence and calculate the limit of the sequence.

\textbf{1. Convergence of the sequence}

We have $x_1 > 0$, $x_{n+1} = \frac{1}{p + x_n}$, so by mathematical induction, we have $x_n > 0$. Consider the function $f(x) = \frac{1}{p+x}$, we know that $f^\prime = -\frac{1}{(p+x)^2}$. We have $\forall x>0, \vert f^\prime(x) \vert = \frac{1}{(p+x)^2} < \frac{1}{p^2} < 1$. 

By the convergence of contractions, we know that the sequence converges.

\textbf{2. Calculate the limit of the sequence}

Now we know that the sequence converges, let the limit be $x$, then we have:
\begin{equation}
  \begin{aligned}
    x &= f(x) = \frac{1}{p+x}\\
    \Rightarrow x &= \frac{\sqrt{p^2+4} - p}{2}
  \end{aligned}
\end{equation}

In conclusion, the value of the continued fraction is $\frac{\sqrt{p^2+4} - p}{2}$.

\section*{VII. Bisection Method with Negative Initial Interval}

Similar to \ref{sec::II}, we can prove the relative error at the $n$-th step is $\eta_n \le \frac{\vert (b_0-a_0)2^{-n-1}\vert}{\vert r \vert} $.

So $\eta_n \le \epsilon \Rightarrow n \ge \frac{\log (b_0-a_0) - \log \epsilon - \log \vert r \vert}{\log 2} - 1$, $r$ represents the root of the function.

There is no better way to imporve the inequality, because we don't know more information about the function and root. 

Due to the root $r$ may very close to $0$ and the inequality is hard to compute, so the relative error is not an appropriate measure of the accuracy of the bisection method.

\section*{VII. Newton's Method with Multiple Zeros}

\subsection*{VII-a}
\label{subsec::VII-a}

We first prove that Newton's method converges linearly when the root is multiple. We assume that the root is $x^*$ and the multiplicity of the root is $k$. So we can write $f(x) = (x-x^*)^k g(x), g(x^*) \ne 0$. Let $\varphi(x) = x - \frac{f(x)}{f^\prime(x)}$, then we have:
\begin{equation}
  \varphi^\prime (x) = 1 - \frac{[f^\prime (x)]^2 - f(x)f^{\prime \prime}(x)}{[f^\prime (x)]^2} = \frac{f(x)f^{\prime \prime}(x)}{[f^\prime (x)]^2}
\end{equation}

$\varphi^\prime (x^*) = \frac{k(k-1)}{k^2} = \frac{k-1}{k} < 1$ (L'Hospital's rule). Therefore, when $x$ is close to $x^*$, $\vert \varphi^\prime (x) \vert < 1$, by the convergence of contractions, we know that the Newton's method converges. 

\textbf{The convergence rate of the Newton's method is linear.}

By Newton's method, we have:

\begin{equation}
  \begin{aligned}
    x_{n+1} &= x_n - \frac{f(x_n)}{f^\prime(x_n)} = x_n - \frac{(x_n-x^*)^k g(x_n)}{k(x_n-x^*)^{k-1}g(x_n) + (x_n-x^*)^k g^\prime(x_n)}\\
    &= x_n - \frac{(x_n-x^*) g(x_n)}{k g(x_n) + (x_n-x^*) g^\prime(x_n)} \\ 
    x_{n+1} - x^* &= (x_n - x^*) (1 - \frac{g(x_n)}{k g(x_n) + (x_n-x^*) g^\prime(x_n)}) \\ 
    \lim_{n \rightarrow +\infty}\frac{x_{n+1} - x^*}{x_n - x^*} &= \frac{k-1}{k}
  \end{aligned}
  \label{eq::VII-a}
\end{equation}

By \ref{eq::VII-a}, we can easily know that $\lim_{n \rightarrow +\infty}\vert \frac{x_{n+2} - x_{n+1}}{x_{n+1} - x_{n}} \vert = \frac{k-1}{k}$

Therefore, the Newton's method converges linearly when the root is a multiple root. 

So if we have the points $(x_n, f(x_n))$. We just calculate $\lim_{n \rightarrow +\infty}\vert \frac{x_{n+2} - x_{n+1}}{x_{n+1} - x_{n}} \vert$, if it converges to $\frac{k-1}{k}$, then it's a $k$-th root. If it converges to 0, then it's a single root. 

\subsection*{VII-b}

Similar to \ref{eq::VII-a}, we define $f(x) = (x-r)^k g(x), g(r) \ne 0$, so by \ref{eq::VII-a}, we have:
\begin{equation}
  x_{n+1} - x^* = \frac{(x_n - x^*)^2 g^\prime(x_n)}{k g(x_n) + (x_n-x^*) g^\prime(x_n)}
  \label{eq::VII-b}
\end{equation}

\textbf{Convergence of the new form of Newton's method}

Define $\varphi(x) = x - k \frac{f(x)}{f^\prime(x)}, f(x) = (x-r)^k g(x)$, then by L'Hospital's rule we have:
\begin{equation}
  \varphi^\prime (r) = 1 - k \frac{[f^\prime (r)]^2 - f(r)f^{\prime \prime}(r)}{[f^\prime (r)]^2} = 1-k + k \frac{k(k-1)}{k^2} = 0
\end{equation}

So $\vert \varphi^\prime (x) \vert < 1$ when $x$ is close to $r$, by the convergence of contractions, the iteration converges. 

\textbf{The convergence is quadratic}

By \ref{eq::VII-b}, we have:
\begin{equation}
  \begin{aligned}
    \frac{x_{n+1} - r}{(x_n - r)^2} = \frac{ g^\prime(x_n)}{k g(x_n) + (x_n-r) g^\prime(x_n)}\\
    \Rightarrow \lim_{n \rightarrow +\infty} \frac{x_{n+1} - r}{(x_n - r)^2} = \frac{g^\prime(r)}{k g(r)}
  \end{aligned}
\end{equation}

% ===============================================
\section*{ \center{\normalsize {Acknowledgement}} }
The section title is generated by \textbf{Kimi}, with a little revise. 
\begin{figure}[H]
  \begin{center}
    \includegraphics[width=0.8\textwidth]{./figure/kimi.png}
  \end{center}
\end{figure}

\printbibliography


\end{document}