\documentclass[a4paper]{article}
\usepackage[affil-it]{authblk}
\usepackage{ctex}
\usepackage{amsmath}
\usepackage{amssymb}
\usepackage{amsfonts}
\usepackage{amsthm}
\usepackage{physics}
\usepackage{graphicx}
\usepackage{hyperref}
\usepackage{listings}
\usepackage{color}
\usepackage{float}
\usepackage{subcaption}
\usepackage{caption}
\usepackage{tikz}
\usepackage{pgfplots}
\hypersetup{colorlinks=true, linkcolor=blue, citecolor=blue, urlcolor=blue}


\usepackage{geometry}
\geometry{margin=1.5cm, vmargin={0pt,1cm}}
\setlength{\topmargin}{-1cm}
\setlength{\paperheight}{29.7cm}
\setlength{\textheight}{25.3cm}

\begin{document}

\title{Numerical Analysis homework \# 7}

\author{王昊 Wang Hao 3220104819
  \thanks{Electronic address: \texttt{3220104819@zju.edu.cn}}}
\affil{(Mathematics and Applied Mathematics 2201), Zhejiang University }


\date{Due time: \today}

\maketitle

\section*{I. Modified Euler's Method}

\textbf{a. Stability}

For the equation 
\begin{equation}
    y^{\prime} = \lambda y, 
\end{equation}

we have
\begin{equation}
    y_{n+1} = y_n + h \frac{\lambda y_n + \lambda y_{n+1}}{2}.
\end{equation}

So we have
\begin{equation}
    \tilde{y}_{n+1} = \frac{2 + \lambda h}{2 - \lambda h} \tilde{y}_n \Rightarrow e_{n+1} = \frac{2 + \lambda h}{2 - \lambda h} e_n.
\end{equation}

So when $\abs{\frac{2 + \lambda h}{2 - \lambda h}} < 1$, the method is stable, which means 
\begin{equation}
    \lambda h < 0.
\end{equation}

\textbf{b. Truncation error}

We have 
\begin{equation}
    \begin{aligned}
        R_n &= y(x_{n+1}) - y(x_n) - h \frac{y^{\prime} (x_n) + y^{\prime}(x_{n+1})}{2} \\
        &= y(x_n) + y^{\prime} (x_n) h + \frac{y^{\prime \prime} (x_n) h^2}{2} + \frac{y^{(3)}(x_n) h^3}{6} + O(h^4) - y(x_n) - h\frac{y^{\prime}(x_n) + y^{\prime}(x_n) + y^{\prime \prime}(x_n) h + \frac{y^{(3)}(x_n) h^2}{2} + O(h^3)}{2} \\
        &= - \frac{y^{(3)}(x_n)}{12} h^3 + O(h^4).
    \end{aligned}
\end{equation}

So truncation error is $O(h^3)$, the method is of order 2.

\textbf{c. Error Estimation}

Similiarly, we assume $f(x,y)$ satisfies Lipschitz condition, with Lipschitz constant $K, L$, and define $M = \sup_{x_0 \leq x \leq b} \abs{f(x, y(x))}$, $M_1 = \sup_{x_0 \leq x \leq b} \abs{y^{\prime \prime}} (x, y(x))$.

For $R_n$, we have
\begin{equation}
    \begin{aligned}
        \abs{R_n} &= \abs{- \frac{y^{(3)}(x_n)}{12} h^3 + O(h^4)} \\
        &\leq M_1 h^3.
    \end{aligned}
\end{equation}

Let $e_n = y_n - y(x_n)$, then we have 
\begin{equation}
    \begin{aligned}
        e_{n+1} &= e_n + h \frac{f(x_n,y_n)-f(x_n,y(x_n)) + f(x_{n+1}, y_{n+1}) - f(x_{n+1},y(x_{n+1}))}{2} - R_n \\
        \Rightarrow e_{n+1} &\leq \frac{2 + hL}{2 - hL} e_n + \frac{2}{2 - hL} R \\
        &\leq \qty(\frac{2 + hL}{2 - hL})^2 e_{n-1} + \qty(\frac{2 + hL}{2 - hL}) \frac{2}{2 - hL} R + \frac{2}{2 - hL} R \\
        &\leq  \qty(\frac{2 + hL}{2 - hL})^{n+1} e_0 + \frac{1}{hL} \qty(\qty(\frac{2+hL}{2-hL})^{n+1} - 1) R \\
        &\leq \mathrm{e}^{\frac{2(b-x_0)L}{2-hL}} e_0 + \frac{h^2 M_1}{L} \qty(\mathrm{e}^{\frac{2(b-x_0)L}{2-hL}} - 1) .
    \end{aligned}
\end{equation}


\section*{II. Forth-order Runge-Kutta method}

\textbf{a. Stability}

We have 
\begin{equation}
    k_1 = \lambda y_n, \quad k_2 = \qty(\lambda + \frac{h}{3} \lambda^2) y_n, \quad k_3 = \qty(\lambda + \frac{2}{3} h \lambda^2 + \frac{h^2}{3} \lambda^3)y_n, \quad k_4 = \qty(\lambda + h \lambda^2 + \frac{h^2}{3}\lambda^3 + \frac{h^3}{3}\lambda^4)y_n. 
\end{equation}

So we have
\begin{equation}
    y_{n+1} = \qty(1 + h \lambda + \frac{h^2}{2} \lambda^2 + \frac{h^3}{6} \lambda^3 + \frac{h^4}{24} \lambda^4) y_n.
\end{equation}

Let $\overline{h} = \lambda h$, then we have 
\begin{equation}
    e_{n+1} = \\qty(1+ \overline{h} + \frac{\overline{h}^2}{2} + \frac{\overline{h}^3}{6} + \frac{\overline{h}^4}{24}) e_n.
\end{equation}

We only need
\begin{equation}
    \abs{1+ \overline{h} + \frac{\overline{h}^2}{2} + \frac{\overline{h}^3}{6} + \frac{\overline{h}^4}{24}} < 1.
\end{equation}


\textbf{b. Error Estimation}

We have 
\begin{equation}
    \begin{aligned}
        k_1 &= y(x_n) \\
        k_2 &= f(x_n, y(x_n)) + f_x \frac{1}{3}h + f_y \frac{1}{3} h y(x_n) + f_{xx} \frac{h^2}{18} + f_{xy} \frac{h^2}{9} y(x_n) + f_{yy} \frac{h^2}{18} y^2(x_n) \\ 
            &+ f_{xxx} \frac{h^3}{162} + f_{xxy} \frac{h^3}{54} y(x_n) + f_{xyy} \frac{h^3}{54} y^2(x_n) + f_{yyy} \frac{h^3}{162} y^3(x_n) + O(h^4) \\
        k_3 &= f(x_n, y(x_n)) + f_x \frac{2}{3}h + f_y (hk_2 - \frac{1}{3}hk_1) + f_{xx} \frac{2h^2}{9} + f_{xy}  \frac{2h}{3} (hk_2 - \frac{1}{3}hk_1) + f_{yy} \frac{(hk_2 - \frac{1}{3}hk_1)^2}{2} \\
        &+ f_{xxx} \frac{4h^3}{81} + f_{xxy} \frac{2h^2}{9} (hk_2 - \frac{1}{3}hk_1) + f_{xyy} \frac{h}{3} (hk_2 - \frac{1}{3}hk_1)^2 + f_{yyy} \frac{(hk_2 - \frac{1}{3}hk_1)^3}{6} + O(h^4) \\
        k_4 &= f(x_n, y(x_n)) + f_x h + f_y (hk_1 - hk_2+hk_3) + f_{xx} \frac{h^2}{2} + f_{xy} h (hk_1 - hk_2+hk_3) + f_{yy} \frac{(hk_1 - hk_2+hk_3)^2}{2} \\
        &+ f_{xxx} \frac{h^3}{6} + f_{xxy} \frac{h^2}{2} (hk_1 - hk_2+hk_3) + f_{xyy} \frac{h}{2} (hk_1 - hk_2+hk_3)^2 + f_{yyy} \frac{(hk_1 - hk_2+hk_3)^3}{6} + O(h^4).
    \end{aligned}
\end{equation}

So we have 
\begin{equation}
    R_n = y(x_{n+1}) - y(x_n) - \frac{h}{8} (k_1 + 3k_2 + 3k_3 + k_4) = O(h^4). 
\end{equation}

So we need to expand more terms to get the accurate error estimation, and we can know that the final result is $R_n = O(h^5)$, so this method is of order 4.


\section*{III. Two Step Method}

\textbf{a. Truncation Error}

We have 
\begin{equation}
    \begin{aligned}
        R_n &= y(x_{n+2}) - (1+\alpha) y(x_{n+1}) + \alpha y(x_n) \\
        &- \frac{h}{12} [(5+\alpha) y^{\prime}(x_{n+2}) + 8(1-\alpha) y^{\prime}(x_{n+1}) - (1+5\alpha) y^{\prime}(x_n)] \\
        &= y + 2h y^{\prime} + 2h^2 y^{\prime \prime} + \frac{4}{3} h^3 y^{(3)} + \frac{2}{3} h^4 y^{(4)} + O(h^5) - (1+\alpha) (y + h y^{\prime} + \frac{h^2}{2} y^{\prime \prime} + \frac{h^3}{6} y^{(3)} + \frac{h^4}{24} y^{(4)} + O(h^5)) + \alpha y \\
        &- \frac{h}{12} [(5 + \alpha) (y^{\prime} + 2h y^{\prime \prime} + 2h^2 y^{\prime \prime \prime} + \frac{4}{3} h^3 y^{(4)} + O(h^4)) + 8(1-\alpha) (y^{\prime} + h y^{\prime \prime} + \frac{h^2}{2} y^{\prime \prime \prime} + \frac{h^3}{6} y^{(4)} + O(h^4)) - (1+5\alpha) y^{\prime}] \\
        &= -\frac{1+\alpha}{24} y^{(4)} h^4 + O(h^5).
    \end{aligned}
\end{equation}

When $\alpha = -1$, the truncation error is $O(h^5)$, for $\alpha \ne -1$, the truncation error is $O(h^4)$.

\textbf{b. Stability}

We have 
\begin{equation}
    \phi(x, \overline{h}) = (1 - \frac{5+\alpha}{12}\overline{h}) x^2 - (1+\alpha + \frac{2}{3}(1-\alpha)\overline{h}) x + (\alpha + \frac{1+5\alpha}{12}\overline{h}).
\end{equation}

We need to let all zeros of $\phi(x, \overline{h})$ have $\abs{x} < 1$, define 
\[A = 1 - \frac{5+\alpha}{12}\overline{h},\quad B = -(1+\alpha + \frac{2}{3}(1-\alpha)\overline{h}),\quad C= \alpha + \frac{1+5\alpha}{12}\overline{h},\]
so we have 
\begin{equation}
    x_1 = \frac{-B + \sqrt{B^2 - 4AC}}{2A}, x_2 = \frac{-B - \sqrt{B^2 - 4AC}}{2A}.
\end{equation}

We need to let $x_1, x_2 < 1$, so we have 
\begin{equation}
    \abs{x_1} = \abs{\frac{-B + \sqrt{B^2 - 4AC}}{2A}}<1, \quad \abs{x_2} = \abs{\frac{-B - \sqrt{B^2 - 4AC}}{2A}}<1.
\end{equation}


\section*{VI. Another Method}

When we put actual values into the expression, we have
\begin{equation}
    g(x,y) = \frac{\partial }{\partial x} f(x, y) + f(x, y) \frac{\partial}{\partial y} f(x, y) = \frac{\mathrm{d}}{\mathrm{d} x} f(x, y(x)) = y^{\prime \prime} (x). 
\end{equation}

We have
\begin{equation}
    \begin{aligned}
        R_n &= y(x_{n+1}) - y(x_n) - h\phi(x_n, y(x_n), h) \\
        &= y(x_n) + f h + \frac{f_x + f f_y}{2} h^2 + \frac{f_{xx} + f_xf_y +2ff_{xy}+ff_y^2+f^2f_{yy}}{6} h^3 \\
        &+ \frac{f_{xxx} + 3ff_{xxy} + 3f^2 f_{xyy} + f_{xx}f_y + 5ff_{xy}f_y + 3f_x f_{xy} + 3ff_x f_{yy} + f_x f_y^2 + 4f^2 f_y f_{yy} + ff_y^3 + f^3 f_{yyy}}{24}h^4 + O(h^5) \\
        &- y(x_n) - fh - \frac{1}{2} h^2 \qty(f_x + f_{xx} \frac{1}{3}h + f_{xy} \frac{1}{3}hf + f_{xxx} \frac{1}{18}h^2 + f_{xxy}\frac{1}{9}h^2f + f_{xyy} \frac{1}{18}h^2f^2 + O(h^3)) \\ 
        &- \frac{1}{2} h^2 \qty(f + f_x \frac{1}{3}h + f_y \frac{1}{3}hf + f_{xx} \frac{1}{18}h^2 + f_{xy} \frac{1}{9}h^2f + f_{yy} \frac{1}{18}h^2f^2 + O(h^3)) \\ 
        & \qty(f_y + f_{xy} \frac{1}{3}h + f_{yy} \frac{1}{3}hf + f_{xxy} \frac{1}{18}h^2 + f_{xyy} \frac{1}{9}h^2f + f_{yyy} \frac{1}{18}h^2f^2 + O(h^3)) \\
        &= O(h^4). 
    \end{aligned}
\end{equation}

So this method is of order 3.

\end{document}