\documentclass[a4paper]{article}
\usepackage{ctex}
\usepackage[affil-it]{authblk}
\usepackage{float}
\usepackage{amsmath, amssymb, amsthm, amsfonts}
% \usepackage[backend=bibtex,style=numeric]{biblatex}
\usepackage{graphicx}
\usepackage{hyperref}
\usepackage{physics}
% 设置超链接样式
\hypersetup{
	colorlinks=true,
	linkcolor=blue,
	citecolor=blue,
	urlcolor=blue,
}

\usepackage{listings}
\usepackage{color}

\definecolor{mygray}{rgb}{0.4,0.4,0.4}
\definecolor{mygreen}{rgb}{0,0.8,0.6}
\definecolor{myorange}{rgb}{1.0,0.4,0}

\lstset{
	basicstyle=\footnotesize\sffamily\color{black},
	breaklines=true,
	commentstyle=\color{mygray},
	frame=single,
	numbers=left,
	numbersep=5pt,
	numberstyle=\tiny\color{mygray},
	keywordstyle=\color{mygreen},
	showspaces=false,
	showstringspaces=false,
	stringstyle=\color{myorange},
	tabsize=2
}

\usepackage{geometry}
\geometry{margin=1.5cm, vmargin={0pt,1cm}}
\setlength{\topmargin}{-1cm}
\setlength{\paperheight}{29.7cm}
\setlength{\textheight}{25.3cm}

\usepackage[titletoc,title]{appendix}
% \usepackage{subcaption}
\usepackage{subfigure}


\begin{document}
% =================================================
\title{Numerical Analysis homework \# 2}

\author{王昊 Wang Hao 3220104819
  \thanks{Electronic address: \texttt{3220104819@zju.edu.cn}}}
\affil{(Mathematics and Applied Mathematics 2201), Zhejiang University }


\date{Due time: \today}

\maketitle

\section*{A. Implement of Newton formula}

To implement the Newton formula in a subroutime that produces the value of the interpolation polynomial $p_n(f;x_0,\cdots,x_n;x)$, we first design the \textbf{NewtonPolynomial} class in C++, it's based on the following formula:
\begin{equation}
	\begin{aligned}
		p_n(f;x_0,\cdots,x_n;x) &= f[x_0] + f[x_0,x_1](x-x_0) + \cdots + f[x_0,\cdots,x_n](x-x_0)\cdots(x-x_{n-1}) \\
		&= f[x_0] + (x-x_0)\qty(f[x_0,x_1] + (x-x_1)\qty(f[x_0,x_1,x_2] + \cdots + (x-x_{n-2})f[x_0,\cdots,x_n])). 
	\end{aligned}
\end{equation}
More details hidden, see the doxygen document and the source code. 

Next we design the implement of the \textbf{Newton} interpolation, which is defined as follows:
\begin{lstlisting}[language=C++]
class Newton : public PolyInterpolation
{
public:
    Newton(const std::vector<double> &xData, const std::vector<double> &yData); // constructor
    Newton(const std::vector<double> &xData, FuncPtr f); // constructor
    NewtonPoly interpolate(const std::vector<double> &xData, const std::vector<double> &yData); // interpolation
    NewtonPoly add_point(double x, double y); // add a point
    double operator()(double x) const override; // operator
    ~Newton() {}
private:
    std::vector<double> x_lists; // x_i
    std::vector<double> y_lists; // f(x_i)
    FuncPtr f; // (optional) f(x)
    std::vector<std::vector<double>> divided_diff; // divided difference table
    NewtonPoly m_poly; // Newton polynomial
};
\end{lstlisting}

To make the code more easy to use, I design two constructors, one is to input the data points $(x_i, f(x_i))$, the other is to input the function $f(x)$ and the data points $x_i$. 

The \textbf{interpolate} function is to calculate the divided difference table and return the Newton polynomial, which will be done automatically in the constructor. 

The \textbf{add\_point} function is to add a new point $(x, f(x))$ to the data points. 

The \textbf{operator()} function is to calculate the value of the Newton polynomial at $x$. The destructor is empty.

So to use the Newton interpolation, we can just do the following:
\begin{lstlisting}[language=C++]
	std::vector<double> xData = {0, 1, 2, 3, 4};
	std::vector<double> yData = {0, 1, 4, 9, 16};
	Newton newton(xData, yData); // finish, the Newton polynomial is stored in newton.m_poly
\end{lstlisting}
Or we can use the following:
\begin{lstlisting}[language=C++]
	std::vector<double> xData = {0, 1, 2, 3, 4};
	double f(double x) { return x * x; }
	Newton newton(xData, f); // finish, the Newton polynomial is stored in newton.m_poly
\end{lstlisting}

More functions used in other questions are documented in the doxygen document. 

The Newton interpolation is heriated from the \textbf{PolyInterpolation} class, which is defined as follows:
\begin{lstlisting}[language=C++]
class PolyInterpolation
{
public:
	virtual double operator()(double x) const = 0;
	virtual ~PolyInterpolation() {}
};
\end{lstlisting}


\section*{B. Question B}

In this question, we will use the Newton interpolation to interpolate the function $f(x) = \frac{1}{1+x^2}$ by the points $x_i = -5 + 5\frac{i}{n},~ i=0,\cdots,n$, and illustrate the Runge phenomenon. The source code is shown in the appendix \ref{appendices::QB}. And the figures are shown below.
\begin{figure}[ht]
    \centering
    \subfigure[$n=2$]
	{
        \includegraphics[width=0.45\textwidth]{./figures/B_Newton2.png}
    }
    \subfigure[$n=4$]
	{
        \includegraphics[width=0.45\textwidth]{./figures/B_Newton4.png}
    }
    \subfigure[$n=6$]
	{
        \includegraphics[width=0.45\textwidth]{./figures/B_Newton6.png}
    }
    \subfigure[$n=8$]
	{
        \includegraphics[width=0.45\textwidth]{./figures/B_Newton8.png}
    }
    \caption{Question B: Runge phenomenon}
    \label{fig::questionB}
\end{figure}

We can see that when $n$ increases, the Runge phenomenon becomes more and more serious, so this method can not be used to interpolate the function $f(x) = \frac{1}{1+x^2}$.

\section*{C. Question C}

In this section, also the function $f(x) = \frac{1}{1+25x^2},~x\in[-1,1]$, but not like the previous question, we will use the Chebyshev nodes to interpolate the function, which means
\begin{equation}
	x_i = \cos (\frac{2i+1}{2n}\pi),~i=0,\cdots,n-1.
\end{equation}
The source code is shown in the appendix \ref{appendices::QC}. And the figures are shown below.
\begin{figure}[ht]
    \centering
    \subfigure[$n=5$]
	{
        \includegraphics[width=0.45\textwidth]{./figures/C_Newton5.png}
    }
    \subfigure[$n=10$]
	{
        \includegraphics[width=0.45\textwidth]{./figures/C_Newton10.png}
    }
    \subfigure[$n=15$]
	{
        \includegraphics[width=0.45\textwidth]{./figures/C_Newton15.png}
    }
    \subfigure[$n=20$]
	{
        \includegraphics[width=0.45\textwidth]{./figures/C_Newton20.png}
    }
    \caption{Question C: Chebyshev nodes}
    \label{fig::questionC}
\end{figure}

We can see that there is no such wide oscillation as the Runge phenomenon, and the interpolation is much better than the previous question.


\section*{D. Question D}

In this quesion, we will the Hermite interpolation to interpolate the function given in the table. The detailed document of this class will be shown in the doxygen document. The source code of this question is shown in the appendix \ref{appendices::QD}. Here I just introduce how to use this class.
\begin{lstlisting}[language=C++]
    std::vector<double> xData = {0, 3, 5, 8, 13, 0, 3, 5, 8, 13};
    std::vector<double> yData = {0, 225, 383, 623, 993, 75, 77, 80, 74, 72};
    std::vector<int> nData = {0, 0, 0, 0, 0, 1, 1, 1, 1, 1};
    Hermite hermite(xData, yData, nData); // finish, the Hermite polynomial is stored in hermite.m_poly
\end{lstlisting}
We just need to input the data points $(x_i, f^{(n)}(x_i), n)$, and the class will finish the interpolation.

Now, use $H(x)$ to represent the Hermite polynomial.

(a) After the interpolation, we can get the Hermite polynomial, and we can calculate the value of the polynomial at $x=10$, which is
\begin{equation}
	H(10) = 742.503.
\end{equation}

(b) We just need to find the maximum value of $H^{\prime}(x),~x\in[0,13]$, or just find the value that $H^{\prime}(x) > 81$ is enough. 

By converting the Hermite polynomial to the normal polynomial, and use the Newton method to find root, we can get that
\begin{equation}
	\max_{x\in[0,13]} H^{\prime}(x) = H^{\prime}(12.3718) = 119.417 > 81. \\
\end{equation}

Or we just search by adding $0.01$ to $x$ until $H^{\prime}(x) > 81$, and we can get that
\begin{equation}
	H^{\prime}(5.92) = 81.0073.\\
\end{equation}

Both of them give the same result, that is there is one time the car exceeds the speed limit. 

The figures are shown below.
\begin{figure}[ht]
	\centering
	\subfigure[$H(x)$]
	{
		\includegraphics[width=0.45\textwidth]{./figures/D_Hermite.png}
	}
	\subfigure[$H^{\prime}(x)$]
	{
		\includegraphics[width=0.45\textwidth]{./figures/D_Hermite_prime.png}
	}
	\caption{Question D: Hermite interpolation}
	\label{fig::questionD}
\end{figure}

\section*{E. Question E}

In this question, we use the Newton interpolate to interpolate two functions, applying the theory to the practice. The source code is shown in the appendix \ref{appendices::QE}. 

(a) By applying the Newton interpolation to them, we can get the following results:
\begin{figure}
	\centering
	\subfigure[Sp1]
	{
		\includegraphics[width=0.4\textwidth]{./figures/E_Newton1.png}
	}
	\subfigure[Sp2]
	{
		\includegraphics[width=0.4\textwidth]{./figures/E_Newton2.png}
	}
	\caption{Question E: Newton interpolation}
	\label{fig::questionE}
\end{figure}

(b) To determine whether they will die in another 15 days, we just need to find the minimial value of the Newton polynomial in the interval $[28,43]$. By finding the root of the derivative of the Newton polynomial, we can get
\begin{equation}
	\begin{aligned}
		\min_{x\in[28,43]} p_1(x) &= p_1(28) = 28.7 > 0, \\
		\min_{x\in[28,43]} p_2(x) &= p_2(28) = 8.89 > 0. \\
	\end{aligned}
\end{equation}
Which means they will not die in another 15 days.

But this is surely not the best way to determine this, because we can find that $p_1(1.95643) = -29.2105 < 0$, which means Sp1 will die just at the early begining. $p_2(x)$ is always positive as $x>0$. 

\section*{F. Question F}

In this question, we use the Bezier curves to plot the heart 
\begin{equation}
	x^2 + (\frac{3}{2}y - \sqrt{\abs{x}})^2 = 3.
\end{equation}

The main idea is to write
\begin{equation}
	\begin{aligned}
		x = x(t), \\
		y = y(t). \\
	\end{aligned}
\end{equation}
And use the Bezier curves to interpolate $x(t)$ and $y(t)$ separately. Therefore, the absolute value of $\gamma^{\prime}(t)$ in the algorithm is not necessary, we can multiply it by a constant to make it easier to calculate. Source code is shown in the appendix \ref{appendices::QF}. The figure is shown below.
\begin{figure}[H]
	\centering
	\subfigure[$m=10$]
	{
		\includegraphics[width=0.4\textwidth]{./figures/F_Bezier_10.png}
	}
	\subfigure[$m=40$]
	{
		\includegraphics[width=0.4\textwidth]{./figures/F_Bezier_40.png}
	}
\end{figure}
\begin{figure}[H]
	\centering
	\includegraphics[scale=0.8]{./figures/F_Bezier_160.png}
	\caption{Question F: Bezier curves}
\end{figure}

Surely when the absolute value of $\gamma^{\prime}(t)$ is large, the Bezier curves is not a simple curve, it intersects itself. So the constant we multiply is important.

\begin{appendices}
	\section{Question B}
	\label{appendices::QB}
	\lstinputlisting[language=C++]{B.cc}

	\section{Question C}
	\label{appendices::QC}
	\lstinputlisting[language=C++]{C.cc}

	\section{Question D}
	\label{appendices::QD}
	\lstinputlisting[language=C++]{D.cc}

	\section{Question E}
	\label{appendices::QE}
	\lstinputlisting[language=C++]{E.cc}

	\section{Question F}
	\label{appendices::QF}
	\lstinputlisting[language=C++]{F.cc}
\end{appendices}

\end{document}